\documentclass[a4paper]{article}

\usepackage[utf8]{inputenc}
\usepackage[T1]{fontenc}
\usepackage{textcomp}
\usepackage[english]{babel}
\usepackage{amsmath, amssymb}
\usepackage{minted}
\usepackage{caption}


% figure support
\usepackage{import}
\usepackage{xifthen}
\pdfminorversion=7
\usepackage{pdfpages}
\usepackage{transparent}
\newcommand{\incfig}[1]{%
    \def\svgwidth{\columnwidth}
    \import{./figures/}{#1.pdf_tex}
}

\DeclareMathSymbol{\C}{\mathalpha}{AMSb}{"43}
\DeclareMathSymbol{\R}{\mathalpha}{AMSb}{"52}

\pdfsuppresswarningpagegroup=1
\newenvironment{code}{\captionsetup{type=listing}}{}
\title{ME 766 - HW1}
\author{Chaitanya Kumar}
\date{}

\begin{document}
    \maketitle
    \section{Serial and OpenMP Codes} 

    Both methods involve an iterative sum without
    dependencies between different iterations. So,
    just a straightforward tree reduction works.
    A slower alternative for implementing a
    multithreaded sum of this sort would be using
    a critical section.
    \subsection*{Trapezoid Rule}

    \begin{code}
    \inputminted[samepage=false, breaklines, firstline=20, lastline=30]{c}{src/integrate.c}
    \label{lst:trapezoid_code}
    \caption{Code for Trapezoid Method}
    \end{code}

    \subsection*{Monte Carlo Rule}
    In my first attempt, I tried to use \texttt{rand()} and its analogues, i.e. the thread safe ones.
    Upon reading the man pages, I realized that
    these functions are not meant to be used in
    a multithreaded environment. The reentrant
    analogues of the random number generators
    end with a \texttt{\_r}. So, finally I stuck with
    \texttt{drand48\_r()}, which returns a double
    in $[0.0,1.0)$, which is pretty much all we need.

    \begin{code}
    \inputminted[samepage=false, breaklines, firstline=33, lastline=51]{c}{src/integrate.c}
    \label{lst:montecarlo_code}
    \caption{Code for Monte Carlo Method}
    \end{code}  


\section{Convergence Test Results}
    \subsection*{Trapezoid Method}
    \begin{figure}[!h]
        \centering
        \includegraphics[width=0.8\textwidth]{dat/convergence_trapezoid.png}
        \caption{Percent error with respect to analytical result for trapezoid rule for different number of bins in integration interval}
        \label{fig:dat-trapezoid_convergence-png}
    \end{figure}
    \pagebreak
    \subsection*{Monte Carlo Method}
    \begin{figure}[!h]
        \centering
        \includegraphics[width=0.8\textwidth]{dat/convergence_montecarlo.png}
        \caption{Percent error with respect to analytical result for the Monte Carlo code for different number of sample points}
        \label{fig:dat-convergence_montecarlo-png}
    \end{figure}

    \pagebreak

    \section{Timing Study of Multithreaded Implementation}
    Nothing fancy going on here, just took the
    average of 5 runs for each thread team size.
    \subsection*{Trapezoid Method}
    \begin{figure}[!h]
        \centering
        \includegraphics[width=0.6\textwidth]{dat/trapezoid_omp.png}
        \caption{Trapezoid Method for different number of threads, $n = 1\text{e}06$}
        \label{fig:dat-}
    \end{figure}

    \subsection*{Monte Carlo Method}

    \begin{figure}[!h]
        \centering
        \includegraphics[width=0.6\textwidth]{dat/montecarlo_omp.png}
        \caption{Monte Carlo Method for different number of threads, $n = 1\text{e}08$}
        \label{fig:dat-montecarlo_omp-png}
    \end{figure}

%%%%%%%%%%%%%%%%%%%%%%%%%%%%%%%%%%%%%%%%%%%%%%%%%%%

\section*{Appendix}
\subsection*{A. Wall Time Code}
\begin{code}
\inputminted[samepage=false, breaklines]{c}{src/timer.c}
\label{lst:timer_src}
\caption{timer.c}
\end{code}      

\subsection*{B. Benchmarking Code}
I guess calling this benchmarking is being generous, but here's the
code anyway.
\begin{code}
\inputminted[samepage=false, breaklines, linenos]{c}{src/testing.c}
\label{lst:testing_src}
\caption{testing.c}
\end{code}      

\subsection*{C. Driver Code}
\begin{code}
\inputminted[samepage=false, breaklines, linenos]{c}{src/main.c}
\label{lst:main_c}
\caption{main.c}
\end{code}      

\subsection*{D. makefile}
\begin{code}
\inputminted[samepage=false, breaklines]{make}{src/makefile}
\label{lst:makefile}
\caption{makefile}
\end{code}      
\end{document}
